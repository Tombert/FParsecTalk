\documentclass{beamer}
\usepackage{default}
\usepackage{listings}
\usepackage{graphicx}
% colors (theme=...): red (default), blue, cyan, orange, green
\mode<presentation>
\usecolortheme{wolverine}
% title and author
\title{FParsec}
\author{Tom Gebert}

% document
\begin{document}
%\titlegraphic{\includegraphics[scale=.1]{vim.png}}
\frame{\titlepage}
\begin{frame} {Parsing}
There is about four hundred billion different ways to parse things, and philosophies that follow from each. 
\begin{itemize}
    \item Pros and cons to each
    \item Different levels of difficulty
    \item Different performance characteristics
\end{itemize}
  
\end{frame}
\begin{frame} {Why is this an important discussion?}
    
    \begin{itemize}
    	\item Parsing logic tends to be one-off. 
    	\item Simple things end up becoming complex. \begin{itemize}
    		\item See: MEGA If-renderer
    	\end{itemize}
    	\item If it's not planned correctly, reuse becomes difficult or impossible. 
    \end{itemize}
\end{frame}
\begin{frame}{What exactly is a Parser Combinator?}
	A parser combinator is function that parses one thing (e.g. a comma)
	\begin{itemize}
	    \item This function can be composed with another function that parses another thing (e.g. a double-quote)
	    \item These functions can be nested and composed arbitrarily and infinitely. 
	\end{itemize}
\end{frame}
\begin{frame}{What is a FParsec?}
   FParsec is a port of the Parsec Haskell library to F\#
   \begin{itemize}
   	 \item Follows F\# conventions a bit more than Haskell.
   	 \item Built-in support for strings and streams. 
   	 \item Very good documentation 
   	 \item Performance is generally ok.\begin{itemize}
   	     \item Performance is quite good for parses that aren't recursively deep.
   	     \item Potentially exponential time, though FParsec combats this by aggressively memoizing
   	 \end{itemize}
   \end{itemize}
\end{frame}
\begin{frame}{What is a FParsec?}
DEMO
\end{frame}


\end{document}
